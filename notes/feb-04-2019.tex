\documentclass[11pt]{article}
\usepackage{graphicx}
\usepackage{amssymb}
%\usepackage[nomarkers]{endfloat}
\usepackage{natbib}
\usepackage{setspace}
\usepackage{wasysym}
\usepackage{wrapfig}

\textwidth = 7 in
\textheight = 9 in
\oddsidemargin = -0.5 in
\evensidemargin = -0.5 in
\topmargin = 0.0 in
\headheight = 0.0 in
\headsep = 0.0 in
\parskip = 0.1in
\parindent = 0in
\renewcommand{\Pr}{\mathbb{P}}
\usepackage{url}
\newcommand{\href}[2]{\url{#2}}
%\usepackage{hyperref}
%\hypersetup{backref,   linkcolor=blue, citecolor=red, colorlinks=false, hyperindex=true}
\begin{document}
notes from Monday, Feb.~4, 2019 \\
\subsection*{Estimating allele freq from genotypes of mom + one child families}
The data: $n$ families that have a genotype of the mother and one child.
For each family we have a pair of observations, a mother's genotype and a child's genotype.
So $x_i = (m_i, c_i)$ if $c_i$ is the child's genotype and $x_i$ is the pair of genotypes
for family $i$.
By definition the likelihood is the probability of the entire data set: $L(q) = \Pr(X\mid q)$

We have 1 parameter that we want to estimate: $q$, the frequency of the $A$ allele (with $B$ being the other allele). So:
\begin{eqnarray}
  \Pr(g = A) & = & q \\
  \Pr(g = B) & = & 1- q
\end{eqnarray}
where $g$ represents some gene sequence drawn from the ``gene pool.''

Assuming the families are independent:
\begin{eqnarray}
  L(q) = \Pr(X\mid q) & = & \prod_{i=1}^{n}\Pr(x_i) 
\end{eqnarray}

It is tempting to simplify by assuming the genotypes are independent:
\begin{eqnarray}\nonumber
  L(q) & = & \prod_{i=1}^{n}\Pr(m_i)\Pr(c_i) \mbox{, but \bf{this is wrong!} } 
\end{eqnarray}
but this assumes that the child's genotype is independent of the mom's genotype. 
That is clearly not a reasonable assumption.

We need to use conditional probability:
\begin{eqnarray}\nonumber
  L(q) & = & \prod_{i=1}^{n}\left[\Pr\left(m_i\right)\Pr\left(c_i\mid m_i\right)\right]
\end{eqnarray}

The first factor for each family the probability that we'd get mom's genotype, if
the allele frequency of $A$ was $q$.
If we assume mom's 2 alleles are independent draws from the ``gene pool'' we can 
make probabilistic statements about genotypes from the allele frequencies (this 
is the Hardy-Weinberg model).
If $m_i$ is the genotype of the $i$th mother:
\begin{eqnarray}
  \Pr(m_i = AA) & = & q^2 \\
  \Pr(m_i = AB) & = & 2(1-q) q \\
  \Pr(m_i = BB) & = & \left(1 - q\right)^2
\end{eqnarray}

The second factor is the probability of each possible value for $c_i$ conditional on $m_i$ and
$q$.
Mendel tells us these.
It is convenient to put them in a table:\\
\begin{center}
\begin{tabular}{|c|c|c|c|}
\hline
         & $\Pr(c_i = AA\mid m_i)$ & $\Pr(c_i = AB\mid m_i)$ & $\Pr(c_i = BB\mid m_i)$ \\
\hline 
$m_i=AA$ & $q$                     & $(1-q)$                 &  $0$ \\
$m_i=AB$ & $q/2$                     & $1/2$                 &  $(1-q)/2$ \\
$m_i=BB$ & $0$                     & $q$                 &  $(1-q)$ \\
\hline
\end{tabular}
\end{center}
note this table shows just the probability of the child conditional on mom's genotype (and the
assumption that dad's contribution is a random draw from the gene pool).
The middle row is a bit trickier than the others:
\begin{eqnarray}
  \Pr(c_i = AA \mid m_i = AB) & = & \Pr(\mbox{mom gives }A)\Pr(\mbox{dad gives }A) \nonumber \\
  & = & (1/2)(q) = q/2 \\
  \Pr(c_i = AB \mid m_i = AB) & = & \Pr(\mbox{mom gives }A)\Pr(\mbox{dad gives }B) + \Pr(\mbox{mom gives }B)\Pr(\mbox{dad gives }A) \nonumber \\
  & = & (1/2)(1-q) + (1/2)(q) = 1/2
\end{eqnarray}
Note that each row of the conditional probability table sums to 1, because each
row considers the entire sample space of the random event $c_i$.

Making the probability statement for each family, is then just the product of the probability
of mom's genotype and the probability of the child conditional on mom's genotype
(the $\Pr\left(m_i\right)\Pr\left(c_i\mid m_i\right)$ factors in the likelihood eqn).
\begin{center}
\begin{tabular}{l|l|l}
$\Pr(m_i=AA, c_i=AA) = q^3$       & $\Pr(m_i=AA, c_i=AB) = q^2(1-q)$ & $\Pr(m_i=AA, c_i=BB) =  0 $\\
&&\\
$\Pr(m_i=AB, c_i=AA) = q^2(1-q)$  & $\Pr(m_i=AB, c_i=AB) = q(1-q)$   & $\Pr(m_i=AB, c_i=BB) = q(1-q)^2$\\
&&\\
$\Pr(m_i=BB, c_i=AA) = 0$         & $\Pr(m_i=BB, c_i=AB) = q(1-q)^2$    & $\Pr(m_i=BB, c_i=BB) = (1-q)^3$\\
\end{tabular}
\end{center}
It is a bit harder to see, but the whole table sums to 1 (for any value of $q$ in the feasible range
$0\leq q\leq 1$).

If we had a big data set we don't care what order we encounter the families. 
The only thing that matters is the number of families observed in each datatype.

Also note that geneticist like to be terse. 
So we might want to denote a genotype with a count of $B$ alleles (so $AA=0$, $AB=1$, $BB=2$).

See how brief this make our table:
\begin{center}
\begin{tabular}{l|l|l}
$\Pr(m_i=0, c_i=0) = q^3$       & $\Pr(m_i=0, c_i=1) = q^2(1-q)$ & $\Pr(m_i=0, c_i=2) =  0 $\\
&&\\
$\Pr(m_i=1, c_i=0) = q^2(1-q)$  & $\Pr(m_i=1, c_i=1) = q(1-q)$   & $\Pr(m_i=1, c_i=2) = q(1-q)^2$\\
&&\\
$\Pr(m_i=2, c_i=0) = 0$         & $\Pr(m_i=2, c_i=1) = q(1-q)^2$    & $\Pr(m_i=2, c_i=2) = (1-q)^3$\\
\end{tabular}
\end{center}

This let's us summarize the data in nine counts: $[n_{00}, n_{01}, n_{02}, n_{10},\ldots,n_{22}]$
where
$$n = \sum_{a=0}^{2}\sum_{b=0}^{2} n_{ab} $$

This make the likelihood contribution very compact, too:
\begin{eqnarray}\nonumber
  L(q) & = & \prod_{i=1}^{n}\left[\Pr\left(m_i\right)\Pr\left(c_i\mid m_i\right)\right] \\
   & = & \prod_{a=0}^{2}\prod_{b=0}^{2}\left[\Pr\left(m_i=a\right)\Pr\left(c_i=b\mid m_i=a\right)\right]^{n_{ab}}
\end{eqnarray}

products are hard to deal with when finding derivatives, so we can move to the log scale:
\begin{eqnarray}\nonumber
 \ln L(q) & = &  \sum_{a=0}^{2}\sum_{b=0}^{2}\ln\left(\left[\Pr\left(m_i=a\right)\Pr\left(c_i=b\mid m_i=a\right)\right]^{n_{ab}}\right)\\
 & = & \sum_{a=0}^{2}\sum_{b=0}^{2}
 n_{ab}\ln\left(\left[\Pr\left(m_i=a\right)\Pr\left(c_i=b\mid m_i=a\right)\right]}\right) \\
 & = & \sum_{a=0}^{2}\sum_{b=0}^{2}\left(
 n_{ab}\ln\left[\Pr\left(m_i=a\right)\right] + n_{ab}\ln\left[\Pr\left(c_i=b\mid m_i=a\right)\right] \right) \\
\end{eqnarray}

Note that if $n_{02} > 0$ or $n_{20} > 0$, we have an ``impossible family'' (indicating an error in our data or a mutation, which is
disallowed in our model).

Note that there is some sharing of terms in the log-likelihood, if we just look at the first term in each log-likelihood:
\begin{eqnarray}
 \ln L(q) & = & \sum_{a=0}^{2}\sum_{b=0}^{2}
 n_{ab}\ln\left[\Pr\left(m_i=a\right)\right] + \sum_{a=0}^{2}\sum_{b=0}^{2} n_{ab}\ln\left[\Pr\left(c_i=b\mid m_i=a\right)\right] \\
 & = & \sum_{a=0}^{2}
 n_{a}\ln\left[\Pr\left(m_i=a\right)\right] + \sum_{a=0}^{2}\sum_{b=0}^{2} n_{ab}\ln\left[\Pr\left(c_i=b\mid m_i=a\right)\right] 
\end{eqnarray}
where we use define $n_{a} = n_{a0} + n_{a1} + n_{a2}$ to denote a count of all of the families that have a particular maternal
genotype.
The notation allows us to skip the summation over $b$ (our index indicating the child's genotype).

Now we can substitute our probability model into the likelihood.
The algebra in parts:
\begin{eqnarray}
\sum_{a=0}^{2}
 n_{a}\ln\left[\Pr\left(m_i=a\right)\right]  & = & n_0\ln(q^2) + n_1\ln(2q(1-q)) + n_2\ln((1-q)^2) \nonumber \\
& = & 2n_0\ln(q) + n_1\ln(2) + n_1\ln(q) + n_1\ln(1-q)) + 2n_2\ln(1-q) \nonumber\\
& = & \left(2n_0 + n_1\right)\ln(q) + \left(2n_2 + n_1\right) \ln(1-q)+ n_1\ln(2)
\end{eqnarray}

There is less simplification when we substitute for $\sum_{a=0}^{2}\sum_{b=0}^{2} n_{ab}\ln\left[\Pr\left(c_i=b\mid m_i=a\right)\right]$, (which we'll call $z$).
Dropping out the terms with coefficients  ($n_{02}$ and $n_{20}$) we see:
\begin{eqnarray}\nonumber
z   & = & n_{00}\ln(q) + n_{01}\ln(1-q) +
          n_{10}\ln(q/2) + n_{11}\ln(1/2) + n_{12}\ln((1-q)/2) + 
           n_{21}\ln(q) + n_{22}\ln(1-q) \\
    & = & \left(n_{00} + n_{10} +n_{21} \right)\ln(q) + \left(n_{01} + n_{12} +n_{22} \right)\ln(1-q)
     + \left(n_{10} + n_{11} + n_{12} + \right)\ln(1/2) \nonumber\\
    & = & \left(n_{00} + n_{10} +n_{21} \right)\ln(q) + \left(n_{01} + n_{12} +n_{22} \right)\ln(1-q)
     - n_{1}\ln(2) 
\end{eqnarray}

So the full log likelihood collapses to a charming:

\begin{eqnarray}\nonumber
\ln L(q)   & = & \left(2n_0 + n_1\right)\ln(q) + \left(2n_2 + n_1\right) \ln(1-q)+ n_1\ln(2) \ldots \\
   & & + \left(n_{00} + n_{10} +n_{21} \right)\ln(q) + \left(n_{01} + n_{12} +n_{22} \right)\ln(1-q)
     - n_{1}\ln(2)  \nonumber \\
   & = & \left(2n_0 + n_1\right)\ln(q) + \left(2n_2 + n_1\right) \ln(1-q)+   \left(n_{00} + n_{10} +n_{21} \right)\ln(q) + \left(n_{01} + n_{12} +n_{22} \right)\ln(1-q) \nonumber \\
 & = & \left(2n_0 + n_1 + n_{00} + n_{10} +n_{21}\right)\ln(q) + \left(2n_2 + n_1 + n_{01} + n_{12} +n_{22}\right) \ln(1-q)
\end{eqnarray}


\end{document}

