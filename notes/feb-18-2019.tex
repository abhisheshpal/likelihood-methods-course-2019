\documentclass[11pt]{article}
\usepackage{graphicx}
\usepackage{amssymb}
%\usepackage[nomarkers]{endfloat}
\usepackage{natbib}
\usepackage{setspace}
\usepackage{wasysym}
\usepackage{wrapfig}

\textwidth = 7 in
\textheight = 9 in
\oddsidemargin = -0.5 in
\evensidemargin = -0.5 in
\topmargin = 0.0 in
\headheight = 0.0 in
\headsep = 0.0 in
\parskip = 0.1in
\parindent = 0in
\renewcommand{\Pr}{\mathbb{P}}
\usepackage{paralist}
\usepackage{url}
\newcommand{\href}[2]{\url{#2}}
\usepackage{hyperref}
\hypersetup{backref,   linkcolor=blue, citecolor=red, colorlinks=true, hyperindex=true}
\begin{document}
notes from the week of Feb.~18, 2019 \\
\tableofcontents

We worked through Bayes' rule and then talked about the difference
between confidence intervals and Bayesian posterior probability
statements (see \url{http://phylo.bio.ku.edu/biostats/bad_bean_counter.pdf})

\section{Posterior distribution for $\nu$}

\section{Revisiting the time to MRCA from mutation counts}
\subsection{the data}
Now we'll add my sequence to the previous example.
We have a tree with 1 or 2 MRCAs.
Our data is
$X=[3,2,6]$,
so our sample size is $n=3$.

\section{the likelihood}
Given that we have worked on similar problems recently, we probably would reach
for the Poisson likelihood:
\begin{eqnarray}
\Pr(X\mid ut_1, ut_2, ut_3)& = & \prod_{i=1}^{n} \Pr(x_i\mid ut_i)\\
& = & \prod_{i=1}^{n} \frac{(ut_i)^{x_i}e^{-ut_i}}{x_i ! }
\end{eqnarray}
This is not wrong, but it fails to accurately convey the constraints on the parameter.
If you and JKK share a mitochondrial MRCA more recently than either of you do with me, then $t_1=t_2$ and $t_3> t_1$.

We can express those constraints on the parameters mathematically as an expression of the domain.
In this case: 
\begin{eqnarray}
    t_1, t_2, t_3, u & \in &  \mathbb{R}\\
    0 & < u < & \infty\\
    0 & < t_1 < & \infty\\
    t_1 & = & t_2 \\
    t_1 & \leq & t_3
\end{eqnarray}
where the first line says that all four parameters are real numbers (not integers, 
not complex numbers, {\em etc.}), and then the equations and inequalities constrain
the feasible range of them.

This is fine and it works fine.

Sometimes it is easier to reparameterize in a way that makes the parameters more
``orthogonal'' (in some vague sense).
What occurs to me is to use $\omega$ to represent a mutation-rate-scaled waiting
time to the previous coalescence event.
That occurs to me because it is the kind of thing we often do in coalescent theory.
It may not occur to you, and that is OK -- you can get the right answers without
this reparameterization.

Specifically:
\begin{eqnarray}
    \omega_1, \omega_2 & \in &  \mathbb{R}\\
    0 & < \omega_1, \omega_2 < & \infty\\
     ut_1 = ut_2 & = & 2\omega_1  \\
    ut_3 &= & 2\omega_2 + \omega_1
\end{eqnarray}


\begin{eqnarray}
\Pr(X\mid \omega_1,\omega_2)& = & \prod_{i=1}^{n} \Pr(x_i\mid \omega_1,\omega_2)\\
& = &
 \frac{(2\omega_1)^2e^{-\omega_1}(2\omega_1)^3e^{-\omega_1}\left(\omega_1+2\omega_2\right)^6e^{-(\omega_1+2\omega_2)}}{2!3!6!} \\
& = & \frac{2^2 2^3}{2!3!6!}\left[\omega_1^5e^{-2\omega_1}\right]
\left[\left(\omega_1+2\omega_2\right)^6e^{-(\omega_1+2\omega_2)}\right] \\
\ln L & = & 5\ln(2)-\ln(2!3!6!)+ \left[5 \ln(\omega_1) -2\omega_1\right] +
\left[6\ln\left(\omega_1+2\omega_2\right) -\omega_1- 2\omega_2)\right] 
\end{eqnarray}
where the [] braces just help us see that the likelihood is coming from 2 paths in the tree: the one between JKK and you $x_1+x_2 = 5$ and the one that says that MTH
is 6 mutations away from that common ancestor $x_3 = 6$

You can derive how $\hat{\omega_1}=2.5$ based soley on $x_1+x_2 = 5$.
However it is not clear if that is actually the maximum likelihood
if we use all the data (and we should always use all of the data).

In this parameterization $\omega_2$ does not affect the probability of $x_1, x_2$ at 
all. 
But $\omega_2$ and $\omega_1$ both appear in the log likelihood from $x_3$.
So if the MLE of $\omega_1$ from $x_3$ does not agree with 2.5, then we 
    will have some tension in the information provided by $x_1, x_2$
    and $x_3$ about what $\omega_1$ should be and the estimate will be 
    some compromise.

\end{document}

