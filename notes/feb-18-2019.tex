\documentclass[11pt]{article}
\usepackage{graphicx}
\usepackage{amssymb}
%\usepackage[nomarkers]{endfloat}
\usepackage{natbib}
\usepackage{setspace}
\usepackage{wasysym}
\usepackage{wrapfig}

\textwidth = 7 in
\textheight = 9 in
\oddsidemargin = -0.5 in
\evensidemargin = -0.5 in
\topmargin = 0.0 in
\headheight = 0.0 in
\headsep = 0.0 in
\parskip = 0.1in
\parindent = 0in
\renewcommand{\Pr}{\mathbb{P}}
\usepackage{paralist}
\usepackage{url}
\newcommand{\href}[2]{\url{#2}}
\usepackage{hyperref}
\hypersetup{backref,   linkcolor=blue, citecolor=red, colorlinks=true, hyperindex=true}
\begin{document}
notes from the week of Feb.~18, 2019 \\
\tableofcontents

We worked through Bayes' rule and then talked about the difference
between confidence intervals and Bayesian posterior probability
statements (see \url{http://phylo.bio.ku.edu/biostats/bad_bean_counter.pdf})

\section{Revisiting the time to MRCA from mutation counts}
\subsection{the data}
Now we'll add my sequence to the previous example.
We have a tree with 1 or 2 MRCAs.

$X=[3,2,6]$
\begin{eqnarray}
\Pr(X\mid \nu_1,\nu_2)& = & \prod_{i=1}{n} \Pr(x_i\mid, \nu_1,\nu_2)\\
& = &
 \frac{\nu_1^2e^{-\nu_1}\nu_1^3e^{-\nu_1}\left(\nu_1+2\nu_2\right)^6e^{-(\nu_1+2\nu_2)}}{2!3!6!} \\
& = & \frac{1}{2!3!6!}\left[\nu_1^5e^{-2\nu_1}\right]
\left[\left(\nu_1+2\nu_2\right)^6e^{-(\nu_1+2\nu_2)}\right] \\
\ln L & = & -\ln(2!3!6!)+ \left[5 \ln(\nu_1) -2\nu_1\right] +
\left[6\ln\left(\nu_1+2\nu_2\right) -\nu_1- 2\nu_2)\right] 
\end{eqnarray}
where the [] braces just help us see that the likelihood is coming from 2 paths in the tree: the one between JKK and you $x_1+x_2 = 5$ and the one that says that MTH
is 6 mutations away from that common ancestor $x_3 = 6$

You can derive how $\hat{nu_1}=2.5$ based soley on $x_1+x_2 = 5$.
However it is not clear if that is actually the maximum likelihood
if we use all the data (and we should always use all of the data).

In this parameterization $\nu_2$ does not affect the probability of $x_1, x_2$ at 
all. 
But $\nu_2$ and $\nu_1$ both appear in the log likelihood from $x_3$.
So if the MLE of $\nu_1$ from $x_3$ does not agree with 2.5, then we 
    will have some tension in the information provided by $x_1, x_2$
    and $x_3$ about what $\nu_1$ should be and the estimate will be 
    some compromise.

\end{document}

