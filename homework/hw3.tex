\documentclass[11pt]{article}
\usepackage{graphicx}
\usepackage{amssymb}
%\usepackage[nomarkers]{endfloat}
\usepackage{natbib}
\usepackage{setspace}
\usepackage{wasysym}
\usepackage{wrapfig}

\textwidth = 7 in
\textheight = 9 in
\oddsidemargin = -0.5 in
\evensidemargin = -0.5 in
\topmargin = 0.0 in
\headheight = 0.0 in
\headsep = 0.0 in
\parskip = 0.1in
\parindent = 0in
\renewcommand{\Pr}{\mathbb{P}}
\usepackage{paralist}
\usepackage{url}
\newcommand{\href}[2]{\url{#2}}
\usepackage{hyperref}
\hypersetup{backref,   linkcolor=blue, citecolor=red, colorlinks=true, hyperindex=true}
\begin{document}

\section*{connecting mutation rate to a probability of mutation}
For closely related sequences, we can ignore ``multiple hits'' (two mutations to the same
site that result in the number of differences being less than the number of mutations).

If we have a mutation rate we can use the Poisson where $\lambda = ut$.
\begin{eqnarray*}
    \Pr(\mbox{a difference} \mid u, t) \approx \Pr(>0 \mbox{ mutations} \mid u, t) = p
    & = & 1 - \Pr(0 \mbox{ mutations} \mid u, t) \\
    & = & 1 - e^{- ut}
\end{eqnarray*}



\section*{Estimating $p$ from data}
If we have 2 and 3 mutations on sister branches (same time to the MRCA) and a total of 
$M$ sites.
For $n$ branches with a vector $y$ changes on each, and $k= \sum y$
\begin{eqnarray*}
   L(p) =  \Pr(y_1=2, y_2=3 \mid p)
   & = & \prod_{i=1}^n {M \choose y_i} p^{y_i}(1-p)^{M-y_i} \\ 
   & = & p^{k}(1-p)^{nM-k} \left(\prod_{i=1}^n {M \choose y_i} \right)  \\ 
   \ln L(p) & = & k\ln(p) + (nM - k)\ln(1-p) + \sum_{i=1}^n \ln {M \choose y_i} \\
   \frac{d\ln L(p)}{dp} & = & \frac{k}{p} - \frac{nM - k}{1-p}\\
   \frac{k}{\hat{p}} & = & \frac{nM - k}{1-\hat{p}} \\
   k - k\hat{p} & = & nM\hat{p} - k\hat{p} \\
   \frac{k}{nM} & = & \hat{p}
\end{eqnarray*}


\section*{conditioning on $N$ instead of $t$}
If we want to estimate the population size, we can integrate over $t$:
\begin{eqnarray*}
\Pr(>0 \mbox{ mutations} \mid u, N) & = & \int_0^\infty\left(1 - e^{- ut}\right)\frac{1}{N}e^{-t/N} dt\\
& = & \int_0^\infty\frac{1}{N}e^{-t/N} - \frac{1}{N}e^{- ut-t/N} dt \\
& = & \left.-e^{-t/N}\right|_{t=0}^{t=\infty} - \frac{1}{N}\int_0^\infty e^{-\frac{(Nu + 1)t}{N}} dt \\
& = & \left.-e^{-t/N}\right|_{t=0}^{t=\infty} + \left.\frac{1}{Nu + 1} e^{-\frac{(Nu + 1)t}{N}}\right|_{t=0}^{t=\infty} \\
& = & \left(0 - (-1)\right) + \left(0 - \frac{1}{Nu + 1}\right) \\
& = & 1 - \frac{1}{Nu + 1}
\end{eqnarray*}

Then, we could do something like:
\begin{eqnarray*}
1 - \frac{1}{\widehat{Nu} + 1} & = & \hat{p}\\
\frac{\widehat{Nu}}{\widehat{Nu} + 1} & = & \hat{p}\\
\widehat{Nu} & = & \hat{p}\widehat{Nu} + \hat{p}\\
\widehat{Nu} - \hat{p}\widehat{Nu} & = & \hat{p}\\
\widehat{Nu} & = & \frac{\hat{p}}{1 - \hat{p}} \\
& = & \frac{k/nM}{1 - k/nM} = \frac{k}{nM-k}
\end{eqnarray*}
to convert our MLE of $p$ to an MLE of $Nu$.



\end{document}

